\documentclass{article}
\usepackage[utf8, nocaptions]{vietnam}

\title{Efficient and robust approximate nearest neighbor search using Hierarchical Navigable Small World graphs}
\author{Vu Quang Son}
\date{04/2024}

\begin{document}
\maketitle

\begin{abstract}
    We present a new approach for the approximate K-nearest neighbor search based on navigable small world
graphs with controllable hierarchy (Hierarchical NSW, HNSW). The proposed solution is fully graph-based, without any need for
additional search structures, which are typically used at the coarse search stage of the most proximity graph techniques.
Hierarchical NSW incrementally builds a multi-layer structure consisting from hierarchical set of proximity graphs (layers) for
nested subsets of the stored elements. The maximum layer in which an element is present is selected randomly with an
exponentially decaying probability distribution. This allows producing graphs similar to the previously studied Navigable Small
World (NSW) structures while additionally having the links separated by their characteristic distance scales. Starting search
from the upper layer together with utilizing the scale separation boosts the performance compared to NSW and allows a
logarithmic complexity scaling. Additional employment of a heuristic for selecting proximity graph neighbors significantly
increases performance at high recall and in case of highly clustered data. Performance evaluation has demonstrated that the
proposed general metric space search index is able to strongly outperform previous opensource state-of-the-art vector-only
approaches. Similarity of the algorithm to the skip list structure allows straightforward balanced distributed implementation.
\end{abstract}

\section{Introduction}


\end{document}